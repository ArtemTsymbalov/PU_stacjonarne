\documentclass[12pt, letterpaper, titlepage]{article}
\usepackage[left=3.5cm, right=2.5cm, top=2.5cm, bottom=2.5cm]{geometry}
\usepackage[MeX]{polski}
\usepackage[utf8]{inputenc}
\usepackage{graphicx}
\usepackage{enumerate}
\usepackage{amsmath} %pakiet matematyczny
\usepackage{amssymb} %pakiet dodatkowych symboli
\title{cw2}
\author{Artem Tsymbalov ISI6}
\date{18 Październik 2022}
\begin{document}
\maketitle
\section{Wilk szary}
Wilk szary, wilk (Canis lupus) – gatunek drapieżnego ssaka z rodziny psowatych (Canidae), zamieszkującego lasy, równiny, tereny bagienne oraz góry Eurazji i Ameryki Północnej. Gatunek o skłonnościach terytorialnych.
\subsection{Występowanie i środowisko}
Wilk szary zamieszkuje Azję oraz północne tereny Europy i Ameryki Północnej.
Najwięcej wilków żyje w Kanadzie (50 tys.), w Rosji (30 tys.) i na terenie Alaski (5–7 tys.)[potrzebny przypis]. W Europie najwięcej wilków występuje w Rumunii (ok. 2,3–2,7 tys.) i Hiszpanii (ok. 2 tys.)[15]. Spotykane są też one m.in. na Półwyspie Skandynawskim, Ukrainie, Słowacji i we Włoszech.
\subsection{Ewolucja i systematyka}
Ewolucja wilka jest długa i skomplikowana. Pierwotnie przyjmowano, że psy (Caninae) wywodzą się z rodzaju Tomarctus (ok. 15 mln lat temu), czyli z bocznej gałęzi podrodziny Borophaginae, której wygląd przypominał nieco nowoczesne psy właściwe. Obecnie uważa się, że był to jedynie przykład wczesnej konwergencji oddzielnych linii filogenetycznych. 
\subsubsection{Etymologia}
Canis: łac. canis „pies”.
lupus: łac. lupus „wilk”.
albus: łac. albus „biały”.
\subsubsection{Skamieniałości wilka na terenie Polski}
Na terenie Polski kopalne szczątki przodków współcześnie występującego wilka znaleziono w okolicach Wyżyny Krakowsko-Częstochowskiej, Sudetów, Tatr i na Rzeszowszczyźnie, gdzie znajdowano je w osadach plejstoceńskich wraz z różnymi śladami życia i bytności tam człowieka. Podobnym znaleziskiem mogą pochwalić się wczesnośredniowieczne stanowiska archeologiczne Gniezna, Gdańska i Opola.
\subsection{Charakterystyka}
Samce wilków (basiory) są większe od samic (wadery) o ok. 20–25%. Długość ciała dorosłego wilka wynosi przeciętnie 100–130 cm[9], nie licząc ogona.
Średnia długość ogona (30–50 cm) to ok. 1/3 długości ciała zwierzęcia. Na grzbietowej części ogona (8–10 cm od jego nasady) znajduje się niebieskoczarny gruczoł nadogonowy, tak zwany fiołkowy, którego znaczenie nie jest jeszcze do końca wyjaśnione.
\subsection{Środowisko i zachowanie}
Wilk występuje w lasach, na równinach, pustyniach, w terenach górskich i bagiennych. Jest gatunkiem terytorialnym. Wyznacza rewir, do którego nie dopuszcza osobników nie należących do watahy. W głębi rewiru, w miejscu najbardziej niedostępnym, urządza legowisko. Jest drapieżnikiem i do swojego życia potrzebuje średnio ok. 1,3 kg mięsa (wraz z kośćmi i skórą) dziennie. W naturze żywi się drobnymi zwierzętami (gryzonie, zające, borsuki, ptaki, bezkręgowce), ale także – o ile warunki i liczebność stada na to pozwala – dużymi zwierzętami kopytnymi. Najczęściej poluje na jelenie, nieco rzadziej wybiera sarny, dziki i łosie. W normalnych warunkach duży wilk z ras północnych może zjeść jednorazowo do 10 kg[9], jednak jest to zwykle związane z wcześniejszą, kilkudniową głodówką. Uzupełnieniem jego diety są owoce i runo leśne. Ich system pokarmowy jest dopasowany do diety mięsnej i różni się od tego, który mają psy domowe (nieco krótszy przewód pokarmowy, wyższa kwasowość, wyższa efektywność trawienia).
\subsection{Znaczenie wilków w przyrodzie}
Wilki odgrywają rolę selekcjonera liczebności dużych ssaków roślinożernych, takich jak sarny i jelenie – które bez redukcji ze strony wilków w znaczny sposób niszczyłyby uprawy leśne oraz uprawy rolne[28].

Wilki, mimo że są w stanie zabić osobniki zdrowe i w sile wieku[29], generalnie wybierają na swoją ofiarę osobniki starsze, młode (do 1 roku życia), słabe i chore[30][29][31], tym samym podnoszą średnią kondycję w populacji ofiar, zwaną też „zdrowotnością”
\end{document}